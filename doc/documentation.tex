\documentclass{article}

% Language setting
% Replace `english' with e.g. `spanish' to change the document language
\usepackage[english]{babel}

% Set page size and margins
% Replace `letterpaper' with `a4paper' for UK/EU standard size
\usepackage[letterpaper,top=2cm,bottom=2cm,left=3cm,right=3cm,marginparwidth=1.75cm]{geometry}

% Useful packages
\usepackage{amsmath}
\usepackage{graphicx}
\usepackage[colorlinks=true, allcolors=blue]{hyperref}

\title{Interactive Illustration of \\ the Sampling Properties of \\ Estimators}
\author{Markus M\"o\ss ler}

\begin{document}
	
\maketitle
	
\begin{abstract}
This is the documentation of the implementation of a learning module for an interactive illustration of the sampling properties of estimators
\end{abstract}

\section{Introduction}

This repository contains the implementation of a learning module with interactive illustrations of fundamental statistical concepts and properties.

\section{Goal of the learning module (Why?)}

Understanding the concept and properties of sampling distributions of estimators is one of the most important concepts of statistical inference, i.e, of learning from data about a (population) model of interest, and thus, a fundamental part of empirical studies in social science in general and econometrics in particular. 
%
Practice, however, has shown that students often have difficulty understanding the concept of sampling distributions and their properties. 
%
One potential reason for this is that the sampling properties of estimators are often formulated and analyzed in an abstract way only. 
%
The goal of this learning module is to give a more intuitive understanding of the sampling properties of estimators using interactive illustrations of simulation results. 

An example is to understand the effect of the sample size $n$ on the sampling properties of the sample average $\overline{X}$ as an estimator for the mean $\mu$ of a random variable of interest as stated in the law of large numbers (LLN) and the central limit theorem (CLT). 
%
Using interactive illustrations of simulation results the students can increase the sample size and observe how the sample average gets closer to the mean (LLN) and how the sampling distribution of the standardized statistic of the sample average gets closer to the standard normal distribution (CLT).

\section{Subject of the learning module (What?)}

\subsection{General}

Subject of this learning module are the sampling properties of estimators for different data generating processes (DGPs). 
%
The DGPs are based on a statistical model with a particular parameter of interest, e.g., the mean of a Bernoulli random variable. 
%
The parameter(s) of interest of the DGPs are estimated using a particular estimator, e.g., the sample average. 
%
This learning module shows the effect of changes in the DGP, e.g., increasing the sample size $n$, on the sampling distribution of an estimator. 

Note, the interactive illustration of the sampling properties of the sample average for increasing the sample size $n$, i.e., the large sample properties of the sample mean, is only one subject of interest. 
%
Other subjects are to understand the effect of omitted variable bias (OVB) and heteroskedasticity on the sampling distribution of the OLS estimator in a simple linear regression model.

\subsection{Univariate random variables and sample average}

\subsubsection{Bernoulli distribution and sample average}

% ber_dis_sam_ave

This illustration shows the effect of changing the sample size $n$ and the probability of success $p$ of the Bernoulli distribution on the sampling distribution of the sample average as estimator for mean of the Bernoulli distribution.

\subsubsection{Continuous uniform distribution and sample average}

% con_und_dis_sam_ave

This illustration shows the effect of changing the sample size $n$ and the lower bound $a$ and the upper bound $b$ of the continuous uniform distribution on the sampling distribution of the sample average as estimator for mean of the continuous uniform distribution.

\subsection{Cross Section Linear Regression Model and Ordinary Least Squares}

Illustration of the properties of the OLS estimator to estimate the slope coefficient $\beta_{1}$ of a linear regression model, i.e.,

\begin{align}
	Y_{i} = \beta_{0} + \beta_{1} X_{i} + u_{i}
\end{align}

\subsubsection{Sampling distribution and sample size}

% cs_lin_reg_ols_01

This illustration shows the effect of increasing the sample size $n$ on the sampling distribution of the OLS estimator for the slope coefficient of a simple linear regression model.

\subsubsection{Sample Size and parameterization of the DGP}

% cs_lin_reg_ols_02

This illustration shows the effect of changing the parameters of the DFP, i.e.,  

\begin{itemize}
	\item[1.] changing the sample size $n$,
	\item[2.] changing the variance of $u_{i}$, i.e., $\sigma_{u_{i}}^{2}$, and,
	\item[3.] changing the variance of $X_{i}$, i.e., $\sigma_{X}^{2}$, 
\end{itemize}

on the sampling distribution of the OLS estimator for the slope coefficient of a simple linear regression model.

\subsubsection{Effect of Heteroskedasticity}

% cs_lin_reg_ols_03

\subsubsection{Effect of Omitted Variable Bias}

% cs_lin_reg_ols_04





\section{Method/implementation of the learning module (How?)}

For an interactive and immediate user experience it is useful to separate the simulation study and the interactive presentation of the results. 
%
This two-step procedure allows also a flexible implementation, i.e., the simulation studies can be conducted with any suitable software, e.g., \emph{R}, \emph{python}, etc. and the results can be presented interactively using basic web development languages, i.e., \texttt{.html}, \texttt{.css} and \texttt{.js}.

\emph{Simulation study and illustration/reporting of the results:}

\begin{itemize}
	\item A realization of the DGP is simulated and the values of a given estimator is calculated.
	\time This procedure is repeated $10,000$ times. 
	\item The simulation results, i.e., the sampling distribution of the estimator, are illustrated using  barplots, scatterplots and/or histograms.
	\item Other simulation results can be reported using tables.
	\item The simulation study is performed for different specifications of the statistical model or for different estimators.
	\item For each specification or estimator the results, e.g., figures and/or tables are saved in different \texttt{.svg} and/or \texttt{html} files.
	\item The simulation study is performed using the programming language for statistical computing and graphics \href{https://www.r-project.org/}{\emph{R}}.	
\end{itemize}

\emph{Interactive presentation of the simulation results:}

\begin{itemize}
	\item The results for different specifications, e.g., figures and/or tables, are interactively embedded in a \texttt{.html} file and linked to a slider input.
	\item The effect of different specifications can be studied by changing the slider, i.e., the specification, and thus, the embedded results.
	\item The interactive integration of the illustrations into the \texttt{.html} file is based on javascript.
\end{itemize}

\emph{Additional material, e.g., for explanation purposes:}

\begin{itemize}
	\item Verbal text or mathematical formula inserted in the \texttt{.html} file directly or using ``collapsibles''.
	\item Interactive animation of changes in the slider inputs with audio explanations.
	\begin{itemize}
		\item For each figure one explanation for the figure is added
		\item For each slider and figure one animated explanation can be added
	\end{itemize}	
	\item Any additional material, e.g., video explanations, can be added into the \texttt{html} file interactively or non-interactively.
\end{itemize}

\emph{Integration in the lecture:}

\begin{itemize}
	\item The material of this learning module can be provided on a gradual basis using links to the specific illustrations/sub modules or as a complete course/module with a starting page and links to the sub modules.
	\item The material can be hosted on \emph{GitHub} or on a learning platform such as \emph{ILIAS}. The easiest way to host the material on a learning platform such as \emph{ILIAS} is using a import interface for \texttt{.html} structures. In the case of the learning platform \emph{ILIAS} this procedure is quite easy and flexible.	
\end{itemize}

\emph{Structure of the learning module:}

\begin{itemize}
	\item The learning module contains different sub modules where each sub module has a specific learning goal, e.g., ``understand the effect of the sample size on the sample properties of the sample average to estimate the mean of a Bernoulli distribution''.
	\item The material of a sub module is collected in a sub folder, e.g., \texttt{ber-dis-sam-ave}.
	\item The sub folder contains:
	\begin{itemize}
		\item \texttt{.R} file, e.g., \texttt{ber\_dis\_sam\_ave.R}, with the simulation study and the results stored in the \texttt{figures} and/or \texttt{tables} subdirectory
		\item \texttt{figures} subdirectory with the illustrations of the simulation results
		\item \texttt{audios} subdirectory with the audio text and files for the explanations using interactive animations
		\item \texttt{tables} subdirectory with the reports of the simulation results (optional)
		\item \texttt{.html} file, e.g., \texttt{ber\_dis\_sam\_ave.html}, with the interactive representation of the illustrations
		\item \texttt{myScript.js} with the javascript for the interactive illustrations
		\item \texttt{myStyle.css} with the css styles for the interactive illustrations
		\item Additional assets, e.g.,:
		\begin{itemize}
			\item \texttt{.png} file with a logo for the header of the \texttt{.html} file
			\item ...
		\end{itemize}
	\end{itemize}
\end{itemize}



\bibliographystyle{alpha}
\bibliography{bibliography}
	
\end{document}