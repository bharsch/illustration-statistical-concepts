\documentclass[12pt]{article}

% Language setting
% Replace `english' with e.g. `spanish' to change the document language
\usepackage[english]{babel}

% Set page size and margins
% Replace `letterpaper' with `a4paper' for UK/EU standard size
\usepackage[letterpaper,top=2cm,bottom=2cm,left=3cm,right=3cm,marginparwidth=1.75cm]{geometry}

% Useful packages
\usepackage{amsmath}
\usepackage{graphicx}
\usepackage[colorlinks=true, allcolors=blue]{hyperref}

\usepackage{float}
% See: https://en.wikibooks.org/wiki/LaTeX/Floats,_Figures_and_Captions#Custom_floats (-> Custom floats)
\floatstyle{plaintop}
\newfloat{CodeSnippet}{!hp}{lop}
\floatname{CodeSnippet}{Code Snippet}

\usepackage{fancyvrb}
\usepackage{setspace} % for doublespacing

\title{Interactive Illustration of \\ the Sampling Properties of \\ Estimators}
\author{Markus M\"o\ss ler}

\begin{document}
	
\maketitle
	
\begin{abstract}
This is the documentation of the implementation of a learning module for an interactive illustration of the sampling properties of estimators
\end{abstract}

\doublespacing

\section{Introduction}\label{SecInt}

This repository contains the implementation of a learning module with interactive and animated illustrations of fundamental statistical concepts and properties.

\section{Goal of the learning module (Why?)}\label{SecWhy}

% dgp and sampling distribution
Understanding the concept and properties of sampling distributions of estimators is one of the most important concepts of statistical inference, i.e, of learning from data about the underlying data generating process (DGP) formalized in probabilistic (population) model of interest, and thus, a fundamental part of empirical studies in social science in general and econometrics in particular. 
%
Practice, however, has shown that students often have difficulty understanding the concept of sampling distributions and their properties. 
%
One potential reason for this is that the sampling properties of estimators are often formulated and analyzed in an abstract way only. 
%
The goal of this learning module is to give a more intuitive understanding of the sampling properties of estimators using interactive and animated illustrations of simulation results. 

% example
One example is to understand the effect of increasing the sample size $n$ on the sampling properties of the sample average $\overline{X}$ as an estimator for the mean $\mu$ of a random variable of interest as stated in the law of large numbers (LLN) and the central limit theorem (CLT). 
%
Using interactive illustrations of simulation results the students can increase the sample size and observe how the sample average gets closer to the mean (LLN) and how the sampling distribution of the standardized statistic of the sample average gets closer to the standard normal distribution (CLT). 

% vs. lecture, textbook and chatgpt
The topics, i.e. the sample distribution of estimators, is part of every statistics and econometrics course and can be found in any introductory textbook for this field. 
%
However, by using interactive and animated illustrations the results we aim to provide a deeper and more intuitive understanding of these concepts. 
%
We believe that this kind of understanding is hard to achieve by just attending a \emph{lecture}, reading a \emph{textbook}, or, by chatting with \emph{ChatGPT} about this topic. 
%
Certainly, another way to achieve this is to provide the implementation of the simulation studies or to let the students to implement the simulation studies themselves. 
%
However, this is often too big a hurdle, especially for undergraduate students. 

\section{Subject of the learning module (What?)}\label{SecWha}

\subsection{General}\label{SecWhaGen}

Subject of this learning module are the sampling properties of estimators for different data generating processes (DGPs). 
%
The DGPs are based on a statistical model with a particular parameter of interest, e.g., the mean of a Bernoulli random variable. 
%
The parameter(s) of interest of the DGPs are estimated using a particular estimator, e.g., the sample average. 
%
This learning module shows the effect of changes in the DGP, e.g., increasing the sample size $n$, on the sampling distribution of an estimator. 

Note, the interactive illustration of the sampling properties of the sample average for increasing the sample size $n$, i.e., the large sample properties of the sample mean, is only one subject of interest. 
%
Other subjects are to understand the effect of omitted variable bias (OVB) and heteroskedasticity on the sampling distribution of the OLS estimator in a simple linear regression model.

\subsection{Univariate random variables and sample average}\label{SecWha01}

\emph{Bernoulli distribution and sample average}

% ber_dis_sam_ave

This illustration shows the effect of changing the sample size $n$ and the probability of success $p$ of the Bernoulli distribution on the sampling distribution of the sample average as estimator for mean of the Bernoulli distribution.

...

\emph{Continuous uniform distribution and sample average}

% con_und_dis_sam_ave

This illustration shows the effect of changing the sample size $n$ and the lower bound $a$ and the upper bound $b$ of the continuous uniform distribution on the sampling distribution of the sample average as estimator for mean of the continuous uniform distribution.

...

\subsection{Linear regression model and ordinary least squares}\label{SecWha02}

Illustration of the properties of the OLS estimator to estimate the slope coefficient $\beta_{1}$ of a linear regression model, i.e.,

\begin{align}
	Y_{i} = \beta_{0} + \beta_{1} X_{i} + u_{i}
\end{align}

\emph{Sampling distribution and sample size}

% cs_lin_reg_ols_01

This illustration shows the effect of increasing the sample size $n$ on the sampling distribution of the OLS estimator for the slope coefficient of a simple linear regression model.

...

\emph{Sample Size and parameterization of the DGP}

% cs_lin_reg_ols_02

This illustration shows the effect of changing the parameters of the DFP, i.e.,  

\begin{itemize}
	\item[1.] changing the sample size $n$,
	\item[2.] changing the variance of $u_{i}$, i.e., $\sigma_{u_{i}}^{2}$, and,
	\item[3.] changing the variance of $X_{i}$, i.e., $\sigma_{X}^{2}$, 
\end{itemize}

on the sampling distribution of the OLS estimator for the slope coefficient of a simple linear regression model.

...

\emph{Effect of Heteroskedasticity}

% cs_lin_reg_ols_03

...

\emph{Effect of Omitted Variable Bias}

% cs_lin_reg_ols_04

...

\section{Method of the learning module (How?)}\label{SecHow}

\subsection{General}\label{SecHowGen}

For an interactive and immediate user experience it is useful to separate the simulation study and the presentation of the results. 
%
This two-step procedure allows also a flexible implementation, i.e., the simulation studies can be conducted with any suitable software, e.g., \emph{R}, \emph{python}, etc. and the results can be presented interactively using basic web development languages, i.e., \emph{html}, \emph{css} and \emph{javascript}.

\vspace{1em}
\noindent\emph{Appealing, flexible, interactive and animated presentation using \emph{html}, \emph{css} and \emph{javascript}}

%\emph{Interactive presentation of the simulation results:}
%
%\begin{itemize}
%	\item The results for different specifications, e.g., figures and/or tables, are interactively embedded in a \texttt{.html} file and linked to a slider input.
%	\item The effect of different specifications can be studied by changing the slider, i.e., the specification, and thus, the embedded results.
%	\item The interactive integration of the illustrations into the \texttt{.html} file is based on javascript.
%\end{itemize}
%
%\emph{Additional material, e.g., for explanation purposes:}
%
%\begin{itemize}
%	\item Verbal text or mathematical formula inserted in the \texttt{.html} file directly or using ``collapsibles''.
%	\item Interactive animation of changes in the slider inputs with audio explanations.
%	\begin{itemize}
	%		\item For each figure one explanation for the figure is added
	%		\item For each slider and figure one animated explanation can be added
	%	\end{itemize}	
%	\item Any additional material, e.g., video explanations, can be added into the \texttt{html} file interactively or non-interactively.
%\end{itemize}

While the setup and the implementation of the simulation or other studies depends on the topic of the learning module, the presentation of the results should be based on core web development tools. 
%
The content of each learning module is presented using \emph{html}, the most popular language to build web pages.\footnote{An introduction to the most important web page development languages can be found here: \href{https://www.w3schools.com/}{\texttt{https://www.w3schools.com/}}} 
%
This allows a very flexible presentation of the results. 
%
The styling of the learning module is based on \emph{css}, a popular language for styling web pages, and on \emph{bootstrap}, a popular \emph{css} framework. 
%
The interactive and animated feature of the learning module is based on \emph{javascript}, a popular language for programming web pages. 
%
The \emph{bootstrap} framework with its grid system allows the development of responsive, mobile-first websites with an appealing design. 
%
This enables an appealing experience of the learning also on mobile devices. 

\vspace{1em}
\noindent\emph{Structure of the \emph{html}, \emph{css} and \emph{javascript} files}

What follows in Section \ref{SecHowHtm} to \ref{SecHowHtm} is an explanation of the structure of the \emph{html}, \emph{css} and \emph{javascript} files.  
%
The explanations are based on the illustration of the properties of the sample average as estimator for the mean of a Bernoulli random variable, i.e., based on module \texttt{ber-dis-sam-ave} which has two parameters (the sample size $n$ and the probability of success $p$) and three figures (the bar plot of number of ones and zero, the histogram of sample average and the histogram of the standardized sample average). 
%
Note, the implementation is flexible in the sense that it can be used for any learning module with two parameters and three figures. 

\subsection{Structure of the file with the simulation study}\label{SecHowSim}

%\vspace{1em}
%\noindent\emph{Simulation study and illustration/reporting of the results}

Depending on the subject of the learning module the number of interactive parameters and the number of figures and/or tables for the illustration of the results are chosen. 
%
For the learning module \texttt{ber-dis-sam-ave} there are two interactive parameters, i,e, sample size $n$ and probability of success $p$, and three figures, i.e., the bar plot of number of ones and zero, the histogram of sample average and the histogram of the standardized sample average. 
%
In this setup the variable and fixed inputs of the simulation study can be defined as in \emph{R} Code Snippet \ref{RCodSniSimSet}. 
%
\begin{CodeSnippet}[!hp]
	\centering
	\caption{\emph{R} code snippet simulation setup}
	\normalsize
	\vspace{0.25cm}
	\begin{BVerbatim}
		
		# inputs (variable)
		n.vec <- c(5, 10, 25, 50, 100)
		p.vec <- c(0.2, 0.4, 0.6, 0.8)
		
		# inputs (fixed)
		RR <- 10000
		
	\end{BVerbatim}
	\label{RCodSniSimSet}
\end{CodeSnippet}
%
The simulation study is based on a loop which runs over the variable inputs (here \texttt{n.vec} and \texttt{p.vec}). 
%
For each variable input combination \texttt{RR=10000} realizations of the DGP are simulated and \texttt{RR=10000} estimates are calculated. 
%
Based on the simulation results for \texttt{RR=10000} realizations the illustration of interest is stored as \texttt{figure\_01\_y\_z.svg}, where \texttt{0x} indicates the \texttt{0x}th figure, \texttt{y} indicates the \texttt{y}th index value of the first parameter and \texttt{z} indicates the \texttt{z}th index value of the second parameter. 
%
E.g. for \texttt{ii =  1} and \texttt{jj = 1} the results for the parameters $n=5$ and $p=0.2$ are stored in \texttt{figure\_01\_1\_1.svg} (see also \emph{R} Code Snippet \ref{RCodSniSimImp} below).%\footnote{The barplot only shows the number of zeros and ones of the last realization of the DGP, i.e. only one out of $10000$.} 
%
\begin{CodeSnippet}[!hp]
	\centering
	\caption{\emph{R} code snippet simulation and illustration implementation}
	\scriptsize
	\vspace{0.25cm}
	\begin{BVerbatim}
		
		# simulation/illustration
		
		for (ii in 1:length(n.vec)) {
			
			for (jj in 1:length(p.vec)) {
				
				NN <- n.vec[ii]
				p <- p.vec[jj]
				
				# simulation
				tmp.sim <- Y_bar_ber_sim_fun(RR = RR, NN = NN, p = p)
								
				# plot no 01
				plt.nam <- paste(fig.dir, "figure_01_", ii, "_", jj, ".svg", sep = "")
				svg(plt.nam) 
				
				...
				
				dev.off()		
				
				...
				
			}
			
		}
		
	\end{BVerbatim}
	\label{RCodSniSimImp}
\end{CodeSnippet}
%
Note, for the integration into the \emph{html} file and the animation using \emph{javascirpit} file later the naming of the figure and/or tables is important (see Section \ref{SecHowHtm}).

%A realization of the DGP is simulated and the values of a given estimator is calculated. 
%%
%This procedure is repeated $10,000$ times. 
%%
%The simulation results, i.e., the sampling distribution of the estimator, are illustrated using barplots, scatterplots and/or histograms. 
%%
%Other simulation results can be reported using tables. 
%%
%The simulation study is performed for different specifications of the DGP and/or for different estimators. 
%%
%For each DGP or estimator the results, e.g., figures and/or tables are saved in different \texttt{.svg} and/or \texttt{html} files. 
%%
%Important, the figures and/or tables should follow a certain naming rule, e.g., \texttt{figure\_0x\_y\_z.svg}, where \texttt{0x} indicates the \texttt{0x}th figure, \texttt{y} indicates the \texttt{y}th index value of the first parameter and \texttt{z} indicates the \texttt{z}th index value of the second parameter. For clarity all figures and/or tables are stored in a subfolder, e.g., \text{./figures}. The naming of the figures and/or tables with the index of the parameter is important for the presentation of the figures and/or tables later. 
%%
%The simulation study is performed using the programming language for statistical computing and graphics \href{https://www.r-project.org/}{\emph{R}}. 
%
%Note that depending on the topic of the learning module, different setups and implementations of the simulation or other studies may be appropriate. 

%..................................................

\subsection{Structure of the \emph{html} file}\label{SecHowHtm}

The core for the interactive and animated implementation consists of six components in form of six tags. 
%
These six tags are introduced below. 

%....................................................................................................

\vspace{1em}
\noindent\emph{Tag 1) Integration of the illustrations}

The illustrations are integrated using \emph{img} tags with the \emph{class} attribute \texttt{figureCl}. 
%
The \emph{src} attributes of the \emph{img} tags point to a particular figure, i.e., the illustration of a particular simulation result. 
%
The \emph{id} attributes of the \emph{img} tags are set to, \texttt{figure1Id}, \texttt{figure2Id}, ..., and used to refer to the \emph{src} attribute of the \emph{img} tag using \emph{javascript}. 
%
Code snippet \ref{HtmlCodSniImgTag} below shows the code of the integration of figure 1 based on a DGP with two interactive parameters, i.e., two slider inputs. 
%
\begin{CodeSnippet}[!hp]
	\centering
	\caption{\emph{Html} code snippet for the integration of figure 1 with two slider inputs}
	\footnotesize
	\vspace{0.25cm}
	\begin{BVerbatim}
		
	<img src="./figures/figure_01_1_1.svg" alt=""
	class="figureCl" id="figure1Id" style="max-width: 75%;">
		
	\end{BVerbatim}
	\label{HtmlCodSniImgTag}
\end{CodeSnippet}

%....................................................................................................

\vspace{1em}
\noindent\emph{Tag 2) Distribution of the illustrations across tabs}

The \emph{img} tags with the \emph{class} attribute \texttt{figureCl} are embedded into \emph{div} tags with the \emph{class} attribute \texttt{tabContentL1Cl} to distribute the illustrations across tabs, i.e., one tab for each figure. 
%
The content of the \emph{div} tags, i.e., the tabs, can be displayed or not displayed using the \emph{display} attribute of the \emph{div} tags. 
%
The \emph{id} attributes of the \emph{div} tags for the tabs are set to, \texttt{tabContentL1N1Id}, \texttt{tabContentL1N1Id}, ..., and used to change the \emph{display} property of the \emph{style} attribute of the \emph{div} tags, i.e, to display or not to display the content of the tabs based on \emph{buttons} using \emph{javascript}. 
%
Code snippet \ref{HtmlCodSniDivTabTag} below shows the code of the integration of tab 1. 
%
\begin{CodeSnippet}[!hp]
	\centering
	\caption{\emph{Html} code snippet for integration of tab 1}
	\footnotesize
	\vspace{0.25cm}
	\begin{BVerbatim}
		
	<div id="tabContentL1N2Id" class="tabContentL1Cl">
		
	  % content of the tab, i.e., figure
		
	</div>
		
	\end{BVerbatim}
	\label{HtmlCodSniDivTabTag}
\end{CodeSnippet}

%....................................................................................................

\vspace{1em}
\noindent\emph{Tag 3) Interaction using sliders}

The parametrization of the DGP cintegrationan be changed using \emph{input} tags\footnote{The \emph{bootstrap-slider} library (see \href{https://github.com/seiyria/bootstrap-slider}{\texttt{https://github.com/seiyria/bootstrap-slider}}) is used to display and interact with the learning module using sliders} with the \emph{class} attribute \texttt{sliderCl}. 
%
The values of the \emph{input} tags will be used to change the \emph{src} attribute of the \emph{img} tags, i.e., to change to the illustration of a particular simulation result. 
%
The \emph{id} attributes of the \emph{input} tags are set to, \texttt{slider1Id}, \texttt{slider2Id}, ... and used to interact with particular sliders using \emph{javascript}. 
%
Code snippet \ref{HtmlCodSniInputTag} below shows the code for the integration of slider 1. 
%
\begin{CodeSnippet}[!hp]
	\centering
	\caption{\emph{Html} code snippet for the integration of slider 1}
	\footnotesize
	\vspace{0.25cm}
	\begin{BVerbatim}
		
	<input class="sliderCl" id="slider1Id" type="text"
	data-slider-min="0" data-slider-max="4" data-slider-step="1"
	data-slider-value="2"/>
		
	\end{BVerbatim}
	\label{HtmlCodSniInputTag}
\end{CodeSnippet}

%....................................................................................................

\vspace{1em}
\noindent\emph{Tag 4) Displaying the values of the sliders}

The value of the sliders, i.e., the underlying specification of the DGP, is displayed using \emph{p} tags with the \emph{class} attribute \texttt{sliderValueCl}. 
%
The \emph{id} attribute of the \emph{p} tags are set to, \texttt{sliderValue1Id}, \texttt{sliderValue2Id}, ..., and used to change the \texttt{.innerHTML} property of the \emph{p} tags to display the slider values using \emph{javascript}. 
%
Code snippet \ref{HtmlCodSniPSliderValueTag} below shows code to display the slider value of slider 1. 
%
\begin{CodeSnippet}[!hp]
	\centering
	\caption{\emph{Html} code snippet to display the value of slider 1}
	\footnotesize
	\vspace{0.25cm}
	\begin{BVerbatim}
		
	<p style="font-size: 12pt; text-align: center;">
	  Sample Size \(n\)
	</p>
	<p style="font-size: 12pt; text-align: center"
	class="sliderValueCl" id="sliderValue1Id">
	</p>
		
	\end{BVerbatim}
	\label{HtmlCodSniPSliderValueTag}
\end{CodeSnippet}

%....................................................................................................

\vspace{1em}
\noindent\emph{Tag 5) Explanations of the figures and the effect of the parametrization}

For each figure an overall explanation can be added. 
% 
Furthermore, for each figure and each slider, i.e., each interactive parameter, a explanation of the effect of the parametrization can be added. 
%
All explanations are collected in \emph{div} tags with the \emph{class} attribute \texttt{audioTextCl} and the \emph{id} set to, \texttt{audioTextFigure1OverallId}, \texttt{audioTextFigure2OverallId}, ..., for the explanations of the figures and, \texttt{audioTextFigure1Slider1Id}, \texttt{audioTextFigure1Slider2Id}, ..., \texttt{audioTextFigure2Slider1Id}, \texttt{audioTextFigure2Slider2Id}, ..., for the explanations of the effects of different parametrization of the DGP. 
%
The \emph{id} attributes of the \emph{div} tags are used to refer to specific explanations. 
%
Longer explanations can be distributed across multiple \emph{span} tags inside the \emph{div} tag. 
%
The distribution across multiple \emph{spans} invokes a break when the explanations is read aloud. 
%
Code snippet \ref{HtmlCodSnipDivOverallExplainText} below shows the integration of an overall explanation text for figure 1. 
%
\begin{CodeSnippet}[!hp]
	\centering
	\caption{\emph{Html} code snippet for adding an overall explanation of figure 1}
	\footnotesize
	\vspace{0.25cm}
	\begin{BVerbatim}
		
    <div id="audioTextFigure1OverallId" class="audioTextCl">
	  <span>The figure shows:<br></span>
	  <span>Blablabla... .</span>
	</div>
		
	\end{BVerbatim}
	\label{HtmlCodSnipDivOverallExplainText}
\end{CodeSnippet}
%
Code snippet \ref{HtmlCodSnipDivSliderExplainText} below shows the integration of an explanation of the effect of changing parameter 1, i.e., slider 1, based on figure 1. 
%
\begin{CodeSnippet}[!hp]
	\centering
	\caption{\emph{Html} code snippet for adding an explanation of the effect of changing slider 1 based on figure 1}
	\footnotesize
	\vspace{0.25cm}
	\begin{BVerbatim}
		
    <div id="audioTextFigure01Slider1Id" class="audioTextCl">
	  <span>Blablablab ... .<br></span>
	  <span>Blablablab ... .</span>
	</div>
		
	\end{BVerbatim}
	\label{HtmlCodSnipDivSliderExplainText}
\end{CodeSnippet}

%....................................................................................................

\vspace{1em}
\noindent\emph{Tag 6) Displaying the explanations of the figures and the effects of the parametrization}
%
The explanation of the figures or the effects of the parametrization are displayed using \emph{p} tags with the \emph{class} attribute \texttt{audioShowTextCl}. 
%
The \emph{id} attribute of the \emph{p} tags are set to, \texttt{audioShowTextFigure1Id}, \texttt{audioShowTextFigure2Id}, ..., and used to change the \texttt{.innerHTML} property of the \emph{p} tags to display the explanations using \emph{javascript}. 
%
Code snippet \ref{HtmlCodSniPShowExplanationTag} below shows code to display the explanation for figure 1. 
%
\begin{CodeSnippet}[!hp]
	\centering
	\caption{\emph{Html} code snippet to display the explanation of figure 1}
	\footnotesize
	\vspace{0.25cm}
	\begin{BVerbatim}
		
		<p id="audioShowTextFigure1Id" class="audioShowTextCl"
		style="font-size: 10pt; color: red; font-style: italic; text-align: center; display: none;">
		</p>
		
	\end{BVerbatim}
	\label{HtmlCodSniPShowExplanationTag}
\end{CodeSnippet}

%....................................................................................................

\vspace{1em}
\noindent\emph{Distribution of the content across bootstrap containers}

The content of the learning module is divided across \emph{bootstrap containers}. 
%
The general scheme of the structure is outlined below. 

\begin{itemize}
	\item 1st Container: Header of the module
	\item 2nd Container with three sections
	\begin{itemize}
		\item Topic of the module
		\item Data generating process (DGP)
		\item Estimator and parameter of interest
	\end{itemize}
	\item 3rd Container:
	\begin{itemize}
		\item Parameter panel with the slider to change the parameter of interest
		\item Illustration panel with four tabs for the illustration of
		\begin{itemize}
			\item Sample draw: Particular outcome
			\item Residual: Particular diagnostic
			\item Estimates: Histogram of parameter estimates
			\item Std. Estimates: Histogram of standardized parameter etimates
		\end{itemize}
	\end{itemize}
	\item 4th Container: More Details
	\begin{itemize}
		\item Simulation Exercise: 
	\end{itemize}
	
\end{itemize}

%Other content such as explanations as plain text or formula or as audio files can be added and can be manipulated, e.g. made visible/invisible inside tabs or collapsibles using \emph{javascript}. 



%....................................................................................................

\subsection{Structure of the \emph{javascript} file}\label{SecHowJs}

...

%....................................................................................................

\subsection{Integration in the lecture}

\begin{itemize}
	\item The material of this learning module can be provided on a gradual basis using links to the specific illustrations/sub modules or as a complete course/module with a starting page and links to the sub modules.
	\item The material can be hosted on \emph{GitHub} or on a learning platform such as \emph{ILIAS}. The easiest way to host the material on a learning platform such as \emph{ILIAS} is using a import interface for \texttt{.html} structures. In the case of the learning platform \emph{ILIAS} this procedure is quite easy and flexible.	
\end{itemize}

%\subsection{Structure of the learning module}
%
%\begin{itemize}
%	\item The learning module contains different sub modules where each sub module has a specific learning goal, e.g., ``understand the effect of the sample size on the sample properties of the sample average to estimate the mean of a Bernoulli distribution''.
%	\item The material of a sub module is collected in a sub folder, e.g., \texttt{ber-dis-sam-ave}.
%	\item The sub folder contains:
%	\begin{itemize}
%		\item \texttt{.R} file, e.g., \texttt{ber\_dis\_sam\_ave.R}, with the simulation study and the results stored in the \texttt{figures} and/or \texttt{tables} subdirectory
%		\item \texttt{figures} subdirectory with the illustrations of the simulation results
%		\item \texttt{audios} subdirectory with the audio text and files for the explanations using interactive animations
%		\item \texttt{tables} subdirectory with the reports of the simulation results (optional)
%		\item \texttt{.html} file, e.g., \texttt{ber\_dis\_sam\_ave.html}, with the interactive representation of the illustrations
%		\item \texttt{myScript.js} with the javascript for the interactive illustrations
%		\item \texttt{myStyle.css} with the css styles for the interactive illustrations
%		\item Additional assets, e.g.,:
%		\begin{itemize}
%			\item \texttt{.png} file with a logo for the header of the \texttt{.html} file
%			\item ...
%		\end{itemize}
%	\end{itemize}
%\end{itemize}





\bibliographystyle{alpha}
\bibliography{bibliography}
	
\end{document}